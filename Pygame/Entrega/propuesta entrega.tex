\documentclass[12pt, a4paper]{article}
	\usepackage[utf8]{inputenc}	
	\usepackage[T1]{fontenc}
	\usepackage[spanish,activeacute]{babel}
	\usepackage{color}	
	\usepackage{fancyhdr}	
	\usepackage{lettrine}
	\usepackage{setspace}
	\usepackage{pdflscape}
	\usepackage{mathtools}
	\usepackage{ulem}
	\usepackage{dsfont}	
	\usepackage{multicol}
	\usepackage{multirow}	
	\usepackage{graphicx}
	\usepackage{amsfonts}
	\usepackage[table,xcdraw]{xcolor}
	\usepackage{tikzsymbols}
	\usepackage{pdfpages}
	\usepackage{cancel}
	\usepackage{amssymb}
	\usepackage{amsthm}
	\usepackage{tcolorbox}
%------------------------------APROVECHAMIENTO DEL FOLIO-------------------------
\usepackage{vmargin}
	\setpapersize{A4}
\setmargins{2.5cm}       % margen izquierdo
{1.5cm}                        % margen superior
{16.5cm}                      % anchura del texto
{23.42cm}                    % altura del texto
{10pt}                           % altura de los encabezados
{1cm}                           % espacio entre el texto y los encabezados
{0pt}                             % altura del pie de página
{2cm}                           % espacio entre el texto y el pie de página        
%---------------------------------------------------------------------------------
%--------------------------------CREACION DE NUMERACIONES AUTOMÁTICAS---------------
\theoremstyle{definition}
\newtheorem{definition}{Definición}

\theoremstyle{plain}
\newtheorem{lemma}{Lema}

\theoremstyle{plain}
\newtheorem{theorem}{Teorema}

\theoremstyle{plain}
\newtheorem{prop}{Proposición}
%----------------------------------INICIAMOS EL DOCUMENTO--------------------------------        
\begin{document}
	\title{Propuesta para la práctica de programación concurrente}
	\author{Daniel Saiz Bautista, Javier López Gismeros y Raúl Blanco Martín }
	\date{abril 2022}
	\maketitle
%-------------------------------------COMANDOS DEFINIDOS POR MI-----------------------------
	%\newcommand{\p}[0]{\mbox{PROP}_{SP}} %ejemplo de comando sin atributo
	%\newcommand{\cts}[1]{\mbox{Cts}_{#1}} %ejemplo de comando con atributo
%----------------------------------------------------------------------------------
	%\tableofcontents
	%\newpage
%	\pagestyle{fancy}
%        \setlength\headheight{23pt}
%        \fancyhf{}
%        \lhead{\bfseries} %Texto arriba izquierda
%        \chead{} %Texto arriba centrado
%        \rhead{} %Texto arriba derecha
%        \lfoot{} %Texto abajo izquierda
%        \cfoot{} %Texto abajo centrado
%        \rfoot{\thepage} %Texto abajo derecha (el atributo pone automaticamente el número de página)
%        \renewcommand{\headrulewidth}{0.1pt} %Tamaño de linea superior
%        \renewcommand{\footrulewidth}{0.1pt} %Tamaño de linea inferior
%----------------------TEXTO DEL DOCUMENTO------------
\section*{Programa elegido}
En nuestro grupo hemos tomado la decisión de utilizar la librería \texttt{pygame}. Debido a que el 
juego requiere de concurrencia hemos descartado cualquier juego por turnos. Por ello, vamos 
a implementar un juego de disparos entre dos naves. El jugador que primero alcance a su 
rival ganaría la partida.

\section*{Paralelismo}
Dentro de la implementación usaremos la librería \texttt{multiprocessing} para poder encerrar 
todas las acciones atómicas que puedan generar problemas. 

\section*{Concurrencia}
Implementaremos, como en el ejemplo que vimos en clase, un cliente/servidor de forma 
que el juego pueda usarse bajo los principios de la programación concurrente, por dos jugadores en la misma red. Para ello utilizaremos, de nuevo, la librería \texttt{multiprocessing} en concreto de 
Listener y Client. 

Posteriormente, una vez alcanzado el objetivo de la práctica con cliente/servidor, intentaremos hacer uso de la tecnología mqtt. Para ello haremos uso de \texttt{paho.mqtt.publish} como se ha visto en clase. 


\end{document}